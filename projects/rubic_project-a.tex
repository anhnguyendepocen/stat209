\documentclass[12pt]{article}

\usepackage{fontspec}
\usepackage{geometry}
\usepackage{lastpage}
\usepackage{fancyhdr}
\usepackage{hyperref}

\geometry{top=1in, bottom=1in, left=1in, right=1in, marginparsep=4pt, marginparwidth=1in}

\renewcommand{\headrulewidth}{0pt}
\pagestyle{fancyplain}
\fancyhf{}
\cfoot{\thepage\ of \pageref{LastPage}}

\setlength{\parindent}{0pt}
\setlength{\parskip}{0pt}

% \setromanfont [Ligatures={Common}, Numbers={OldStyle}, Variant=01,
%  BoldFont={LinLibertine_RB.otf},
%  ItalicFont={LinLibertine_RI.otf},
%  BoldItalicFont={LinLibertine_RBI.otf}
%  ]{LinLibertine_R.otf}

\usepackage{tikz}
\def\checkmark{\tikz\fill[scale=0.4](0,.35) -- (.25,0) -- (1,.7) -- (.25,.15) -- cycle;}

\usepackage{xunicode}
\defaultfontfeatures{Mapping=tex-text}

\setromanfont{YaleNew}

\begin{document}

\begin{center}
{\bf Project 01 - Rubric} \\
\end{center}

\bigskip

This rubric will be used to grade your submission for the first
project. Please pay attention to the rubric and make sure you have
followed all of the instructions. Note that the relationship between
points and letter grades may not follow the standard conventions
(that is, 90\% may not be an A-), and generally components will not
receive partial credit.

\bigskip

Name: \underline{\hspace{3cm}} \\  \\
Total Points: \underline{\hspace{1cm}} \textbf{(out of 355pt)} \\
Letter Grade: \underline{\hspace{3cm}}

\bigskip

\textbf{Dataset:} \vspace{6pt}

\underline{\hspace{1cm}} \textbf{(10pt)}
  the dataset was correctly formatted and uploaded to GitHub \\
\underline{\hspace{1cm}} \textbf{(10pt)}
  the dataset contains at least 5 variables and 75 observations \\
\underline{\hspace{1cm}} \textbf{(10pt)}
  the dataset contains at least 2 numeric or time variables \\
\underline{\hspace{1cm}} \textbf{(10pt)}
  the dataset contains at least 1 categorical variable \\
\underline{\hspace{1cm}} \textbf{(10pt)}
  the dataset was collected by the student and not directly available \\
\underline{\hspace{1cm}} \textbf{(10pt)}
  the variables in the dataset following the course's naming conventions \\

\medskip

\textbf{Report (general):} \vspace{6pt}

\underline{\hspace{1cm}} \textbf{(10pt)}
  the report was correctly formatted \\
\underline{\hspace{1cm}} \textbf{(10pt)}
  the report was compiled to html using RMarkdown \\
\underline{\hspace{1cm}} \textbf{(05pt)}
  the student filled in all of the required YAML metadata \\
\underline{\hspace{1cm}} \textbf{(05pt)}
  all of the instructions were removed from the report \\
\underline{\hspace{1cm}} \textbf{(05pt)}
  the dataset is correctly read from GitHub into the report \\
\underline{\hspace{1cm}} \textbf{(20pt)}
  the overview description gives a clearly written description
  of the dataset \\

\medskip

\textbf{Schema:} \vspace{6pt}

\underline{\hspace{1cm}} \textbf{(10pt)}
  the variables described correctly match the dataset presented \\
\underline{\hspace{1cm}} \textbf{(15pt)}
  all of the descriptions fully and accurately describe the variables \\
\underline{\hspace{1cm}} \textbf{(15pt)}
  descriptions include the correct data types for each variable \\
\underline{\hspace{1cm}} \textbf{(10pt)}
  descriptions include units where appropriate \\

\medskip

\textbf{Univariate Analysis:} \vspace{6pt}

\underline{\hspace{1cm}} \textbf{(20pt)}
  all of the variables are included in the report \\
\underline{\hspace{1cm}} \textbf{(20pt)}
  appropriate plots and tables are used in each part of the analysis \\
\underline{\hspace{1cm}} \textbf{(10pt)}
  the number of plots / summaries is not excessive (only 1-2 per variable) \\
\underline{\hspace{1cm}} \textbf{(20pt)}
  the descriptions of the variables are written in full, clear sentences \\
\underline{\hspace{1cm}} \textbf{(20pt)}
  the descriptions match the plots / tables shown \\

\newpage

\textbf{Exploratory Analysis:} \vspace{6pt}

\underline{\hspace{1cm}} \textbf{(25pt)}
  the exploratory analysis includes 2-3 clearly written paragraphs
  describing one or more relationships found in the data \\
\underline{\hspace{1cm}} \textbf{(15pt)}
  1-2 graphics are included to illustrate the relationship \\
\underline{\hspace{1cm}} \textbf{(15pt)}
  the graphics are succesfully integrated into the analysis \\
\underline{\hspace{1cm}} \textbf{(20pt)}
  the student explains any broader take-aways of the analysis
  or suggests next steps for investigating the relationship further
  with new data \\

\medskip

\textbf{Overall:} \vspace{6pt}

\underline{\hspace{1cm}} \textbf{(30pt)}
  the report demonstrates that the student has collected a dataset
  that not only fulfills the literal requirements but also has
  attempted to collect an interesting dataset that is of actual
  interest to them \\



\end{document}





