\documentclass[12pt]{article}

\usepackage{fontspec}
\usepackage{geometry}
\usepackage{lastpage}
\usepackage{fancyhdr}
\usepackage{hyperref}

\geometry{top=1in, bottom=1in, left=1in, right=1in, marginparsep=4pt, marginparwidth=1in}

\renewcommand{\headrulewidth}{0pt}
\pagestyle{fancyplain}
\fancyhf{}
\lfoot{Valid as of 2018-01-11}
\rfoot{page \thepage\ of \pageref{LastPage}}

\setlength{\parindent}{0pt}
\setlength{\parskip}{0pt}

% \setromanfont [Ligatures={Common}, Numbers={OldStyle}, Variant=01,
%  BoldFont={LinLibertine_RB.otf},
%  ItalicFont={LinLibertine_RI.otf},
%  BoldItalicFont={LinLibertine_RBI.otf}
%  ]{LinLibertine_R.otf}

\usepackage{tikz}
\def\checkmark{\tikz\fill[scale=0.4](0,.35) -- (.25,0) -- (1,.7) -- (.25,.15) -- cycle;}

\usepackage{xunicode}
\defaultfontfeatures{Mapping=tex-text}

\setromanfont{YaleNew}

\begin{document}

\begin{center}
{\bf MATH/STAT 209: Introduction to Statistical Modelling, Spring 2018} \\
Tuesday, Thursday 10:30-11:45 \quad JPSN G23\\
Tuesday, Thursday 12:00-13:15 \quad JPSN G28\\
Tuesday, Thursday 15:00-16:15 \quad JPSN G23
\end{center}

\bigskip

\noindent
\begin{tabular}{ l l }
{\bf Instructor:} &  {\bf Taylor Arnold} \\
E-mail: & \href{mailto:tarnold2@richmond.edu}{tarnold2@richmond.edu} \\
Office: & Jepson Hall, Rm 218 \\
Office hours: & Tuesday, Thursday 16:15-17:00
\end{tabular}

\vspace{0.5cm}

\textbf{Description:} \vspace{6pt}

This course broadly covers the entire process of collecting,
cleaning, visualizing, modeling, and presenting datasets. It
has a MATH designation but is not a \textit{mathematics} course.
The focus is on applied statistics and data analysis
rather than a detailed study of symbolic mathematics. By the end
of the semester you will feel confident collecting, analyzing,
and writing about datasets from a variety of fields. You will be
able to use these skills to address data-driven problems in a wide
range of application domains.

\bigskip

\textbf{Computing:} \vspace{6pt}

To facilitate your ability to actually \textit{do} statistics,
most class meetings will involve some form of computing.
No prior programming experience is assumed or required.

\medskip

We will use the \textbf{R} programming environment throughout the
semester. It is freely available for all major operating systems and
is pre-installed on many campus computers. You can download it and
all supporting files for your own machine via these links:
\begin{center}
\url{https://cran.r-project.org/} \\
\url{https://www.rstudio.com/}
\end{center}
The lab computers in Jepson are available and contain all of the
required software. I strongly recommend, however, downloading these
on your own machine so that you will be able to work on assignments
without needing to work only in the computer lab.

\bigskip

\textbf{Course Website:} \vspace{6pt}

All of the materials and assignments for the course will be posted
on the class website:
\begin{quote}
\url{https://statsmaths.github.io/stat209}
\end{quote}
The website contains notes, assignment details, and
supplemental materials. At the end of the semester, this version of
the course will be archived and available for your reference.

\vspace{0.4cm}

\textbf{GitHub:} \vspace{6pt}

Your work for this semester will be submitted through GitHub,
the same platform that hosts our website, using the GitHub
classroom program. You will need to set up a free account, which
we will cover during the first several week of class.

\newpage

\textbf{In-Class Assessments (quizzes):} \vspace{6pt}

On most Tuesdays, there will be a short assessment covering
the material from the prior week. Note that this includes both
a conceptual understanding of the topics covered as well as the
ability to apply these concepts to data with code. I will provide
details on the class website with the exact topics on each assessment.

\bigskip

\textbf{Data Projects:} \vspace{6pt}

While the assessments serve to make sure you are following along
with the general concepts, the core aim of the course is to teach
you how to \textit{apply} statistics to real-world questions. To
this end, we will complete several data-oriented projects.
These projects consist of short written documents that mix code,
graphics, and prose to provide a comprehensive analysis of a data set.

\bigskip

\textbf{Final Grades:} \vspace{6pt}

The in-class assessments, of which there will 12, are graded on
a strictly pass/fail basis. The whole of these are converted into
a letter grade according to how many you have passed:
\begin{itemize}\setlength\itemsep{0em}
\item \textbf{A} \quad \quad \quad Passing 11/12 assessments
\item \textbf{A-/B+} \quad Passing 10/12 assessments
\item \textbf{B} \quad \quad \quad Passing 9/12 assessments
\item \textbf{B-/C+} \quad Passing 8/12 assessments
\item \textbf{C} \quad \quad \quad Passing 7/12 assessments
\item \textbf{D} \quad \quad \quad Passing 6/12 assessments
\end{itemize}
Your projects will receive a letter grade according to a rubric that
I will distribute ahead of the due-date. The final grade will be
determined by weighting the assessments and projects as follows:
\begin{itemize}\setlength\itemsep{0em}
\item \textbf{Projects}: 67\%
\item \textbf{Assessments}: 33\%
\end{itemize}
To pass the course, you must also miss no more than four class meetings.
Attendance requires that you arrive on-time, complete any out of class
assignments for the day, and fully engage with the course material.
Failing to fulfill these attendance requirements may result in a failing
grade for the course.

% To accomplish this goal, this I have implemented an adapted
% standards-based grading system for the course.

% \bigskip

% Labs (homework), reports, quizzes, and exams are graded  In some
% instances, particularly on exams, there will also be a grade of
% of a low-pass (LP). These scores, together with your attendance
% and course engagement, yield a final grade based on the following
% criteria.

% \begin{center}
% \begin{tabular}{p{2.5cm}|p{4cm}|p{4cm}|p{4cm}}
% & \quad \textbf{A} & \quad \textbf{B} & \quad \textbf{C} \\ \hline \hline
% \textbf{Participation} & actively present and engaged in almost all classes & actively present and engaged in most classes & actively present and engaged in most classes (miss at most 5)  \\ \hline
% \textbf{Labs} & passing grade on nearly all labs (missing no more than 2) & passing grade on most labs (missing no more than 4) & passing grade on the majority of labs (missing no more than 6) \\ \hline
% \textbf{Quizzes} & pass at least 7 quizzes & pass at least 5 quizzes & pass at least 3 quizzes \\ \hline
% \textbf{Reports} & pass all three reports & pass all three reports & pass at least two data analyses \\ \hline
% \textbf{Exams} & achieve a high-pass on both exams & achieve one high-pass and one pass on the exams & achieve passing scores on both exams\\ \hline
% \end{tabular}
% \end{center}

% You are allowed one `freebie' on these requirements; for example, you
% can get a B if you fail one data report but fulfill all other
% requirements. Note that failing to meet the requirements for a B will
% result in a grade no higher than a B; that is, you cannot typically
% make-up for performance in one area of the course with good performance
% in another. Failure to meet the requirements for a C may result in a
% failing grade for the term.

% \bigskip

% There is no hard rule for assigning pluses and minuses, but generally an
% A- is given for missing 2-3 requirements of an A and a B+ is given for
% missing 4-5 requirements. Similar rules apply for receiving grades of
% a B- or C+.

\bigskip

\textbf{Class Policies:} \vspace{6pt}

The following class policies address some of the most common
questions and concerns that students have. If anything is
unclear, please feel free to contact me for clarification at
any point in the semester.

\begin{itemize}\setlength\itemsep{0em}
\item \textbf{Academic honesty:} Cheating and plagiarism are grave scholarly
offenses and potential grounds
for expulsion; they are also a major barrier to your intellectual development.
You are expected to familiarize yourself with the entirety of the
University of Richmond’s Honor Code. If you are confused or unsure about
appropriate citation protocol or any other aspect of the Honor code,
please consult me before turning in an assignment.
\item \textbf{Special approval:} If you have special approval forms for extra
time on exams or any other circumstances I should know about, please speak
with me as early as possible so that we can best accommodate your needs.
\item \textbf{Late work:} You are expected to submit all work on-time. Late
reports will not be considered; always hand in something even if it is not perfect.
\item \textbf{Attendance:} You are expected to both attend and participate in most
class meetings. If you must be absent due to illness or other pressing
need, please let me know by email as soon as possible. A habit of arriving
late, failing to participate, or failing to accomplish any out of class assignments
is considered equivalent to an absence.
\item \textbf{Make-up work:} In instances where students have a valid excuse for
missing a quiz, please arrange to meet with me as soon as possible.
\item \textbf{Class conduct:} During class I expect you to refrain from checking
email, being on phones, or working on assignments for other classes.
\item \textbf{Computers:} During programming assignments started in class, I expect you
to use the computers in the lab. This is helpful for several reasons: it
reduces distractions from iMessages and other materials on your laptop;
all of the lab computers are configured using the same software and language
set-up, reducing errors specific to your machine; and, other students and
myself can share the same screen without worrying about modifying something
on your personal machine.
\item \textbf{Office hours}: If you would like to meet during my office hours,
please just come by. No need to schedule an appointment. If you find me in
my office at other times of the week, I am usually glad to meet then as well.
Finally, I am also happy to make appointments outside of my normal
office hours. These appointments are meant for discussing
longer issues that are not appropriate for regular office hours (i.e., asking
for recommendation letters or discussing an extended absence) or for students
who cannot make my normal office hours. Please note that appointments should
be booked at least 24 hours ahead of time.
\item \textbf{Email:} I will also answer questions by email (it can, in fact,
be much faster than scheduling an appointment for small issues). During the
week, I aim to respond within 24 hours, with emails sent over the weekend
responded to by Monday morning. If your question involves code, please attach
your current lab or report as that will expedite my answering your question(s).
\end{itemize}

\bigskip

\textbf{Notice:} \vspace{6pt}

I reserve the right to modify this syllabus, with advanced warning, throughout
the semester. If necessary, I will email the class list and post and updated
version of the document on the course website.

% %%%%%%%%%%%%%%%%%%%%%%%%%%%%%%%%%%%%%%%%%%%%%%%%%%%%%%%%%%%%%%%%%%%%%
% \newpage

% \textbf{Learning Objectives (detail):} \vspace{6pt}

% Below is a more detailed description of each week's material. Note
% that the exact topics, pace, and coverage may change over the course
% of the semester.

% \vspace{1.0cm}

% \def\labelitemi{}
% \def\labelitemii{}

% \underline{Week 01, Reproducible research:}
% In this unit we explore the basics of statistical computing.
% We look at examples and benefits of plain text formats for data,
% code, and analyses.
% \begin{itemize}\setlength\itemsep{0em}
% \item
%   install R, RStudio and user-contributed packages
% \item
%   understand basic version control using the web-based GitHub GUI
% \item
%   create CSV files and read them into R
% \item
%   basic techniques for accessing variables in data objects
% \item
%   selecting variable names
% \item
%   constructing a data dictionary
% \end{itemize}

% \bigskip

% \underline{Week 02, Variable types and numeric summaries:}
% We begin our study of tabular data, with observations stored in rows and
% variables stored in columns. We start by describing the types of
% data that can be stored. Next, methods for summarizing and graphing
% numeric and categorical data are developed.
% \begin{itemize}\setlength\itemsep{0em}
% \item
%   describe and compute means and medians (using R and by hand)
% \item
%   describe and compute quantiles (using R and, for simple cases, by
%   hand)
% \item
%   describe and compute the standard deviation
% \item
%   differentiate between numeric and categorical data
% \item
%   treating numeric data as categorical data
% \item
%   create categorical variables by grouping numeric data
% \end{itemize}

% \bigskip

% \underline{Week 03, The Grammar of Graphics:}
% Building off of our basic plots, here we describe a self-contained
% system for the creation of statistical graphics. Putting the topics
% together allow for the construction of arbitrarily complex
% visualizations of data.
% \begin{itemize}\setlength\itemsep{0em}
% \item
%   the basic data aesthetics (x, y, label, color, size, and alpha)
% \item
%   the \textbf{ggplot2} syntax
% \item
%   mapping variables to aesthetics
% \item
%   setting fixed aesthetics
% \item
%   constructing layers of points, lines
% \end{itemize}

% \bigskip

% \underline{Week 04, Advanced Graphics:}
% Here we build off of the grammar of graphics to include
% other layer types, manual annotations, and building
% professional graphics.
% \begin{itemize}\setlength\itemsep{0em}
% \item
%   describe scales and themes in the grammar of graphics
% \item
%   use manual annotations to give context to graphics
% \item
%   make use of multiple datasets within a single plot
% \item
%   faceting by categorical variables
% \end{itemize}

% \bigskip

% \underline{Week 05, Filtering and Summarizing Data:}
% Often, data is given or found in a different format than is required for
% an analysis. In this unit, we study techniques for filtering and
% restructuring data. One particular focus is the study of how to
% change the \textit{level of analysis} of a data set, such as taking
% data originally about individuals and turning it into a dataset to
% study cities or counties.
% \begin{itemize}\setlength\itemsep{0em}
% \item
%   syntax of the filter command
% \item
%   boolean variables
% \item
%   binary operators: ``and'', ``or''
% \item
%   binary operators: greater than, less than
% \item
%   set containment
% \item
%   random subsets
% \item
%   syntax of the summarize command
% \item
%   counting grouped data
% \end{itemize}

% \bigskip

% \underline{Week 06, Statistical Inference:}
% Given a sample of data taken from a larger population, we can use
% models to estimate how well the sample resembles the entire population.
% This unit covers an introduction to these techniques, known as statistical
% inference. The same techniques can be used to study the outcome of random
% processes.
% \begin{itemize}\setlength\itemsep{0em}
% \item
%   sample and population statistics
% \item
%   independence
% \item
%   standard errors
% \item
%   confidence intervals
% \item
%   t-tests
% \end{itemize}

% \bigskip

% \underline{Week 07, Data Analysis and Review:}
% A review of the prior weeks and applications to a
% new dataset. First midterm on Thursday.

% \bigskip

% \underline{Week 08, Communicating Statistical Results:}
% In this unit we cover how to construct arguments using evidence derived
% from data.
% \begin{itemize}\setlength\itemsep{0em}
% \item
%   deductive versus inductive reasoning
% \item
%   understanding audience (technical vs.~general)
% \item
%   describe exploratory work and hypothesis / thesis generation
% \item
%   describe inferential statistics and hypothesis validation
% \item
%   include graphical annotations
% \end{itemize}

% \bigskip

% \underline{Spring Break}

% \bigskip

% \underline{Week 09, Normalized Data:}
% In this unit, we will explore and implement best practices for collecting and
% organizing data.
% \begin{itemize}\setlength\itemsep{0em}
% \item
%   determining table variables
% \item
%   specifying variable types
% \item
%   consistency standards
% \item
%   ISO date and time standards (ISO 8601)
% \item
%   standards for location data: country codes (ISO 3166), languages (ISO
%   639), currency (ISO 4217), and US FIPS codes
% \item
%   the tidy data model: rows, columns, and tables
% \end{itemize}

% \bigskip

% \underline{Week 10, Joining Relational Data:}
% We also build off of the last week's material by exploring
% concepts of data manipulation and data
% collection to describe methods for working simultaneously with many
% tables linked together by common keys.
% \begin{itemize}\setlength\itemsep{0em}
% \item
%   primary keys
% \item
%   foreign keys
% \item
%   composite keys
% \item
%   inner and outer joins
% \item
%   filtering joins
% \end{itemize}

% \bigskip

% \underline{Week 11, Working with strings:}
% Here, we study techniques for using manipulating data stored as strings. We use the stringi library in R to apply functions using fixed strings as well as a standard for describing patterns called regular expressions. You will become familiar with the following tasks and concepts:
% \begin{itemize}\setlength\itemsep{0em}
% \item detecting substrings
% \item extracting substrings
% \item removing substrings
% \item counting substrings
% \item describing repeating patterns
% \item denoting letters, numbers, and word boundaries
% \item anchoring regular expressions
% \item the UTF-8 encoding
% \item ICU and ISO-639
% \end{itemize}

% \bigskip

% \underline{Week 12, Text mining:}
% This unit builds off of the basic string processing tasks to study
% textual corpora. You will become familiar with the following concepts
% and comfortable applying them to new, raw textual data:
% \begin{itemize}\setlength\itemsep{0em}
% \item tokenization
% \item term-frequency matrices
% \item lemmatization
% \item part-of-speech tags
% \item dependencies
% \item named entities
% \end{itemize}

% \bigskip

% \underline{Week 13, Ethical Guidelines for Statistical Practice:}
% We discuss ethical issues surrounding the practice of data analysis.
% These include:
% \begin{itemize}\setlength\itemsep{0em}
% \item data ownership
% \item informed consent and IRB
% \item data privacy and anonymity
% \item open data
% \item $p$-hacking / data dredging
% \item problematic variables, discrimination, and proxies
% \item algorithmic fairness
% \item confirmation bias
% \end{itemize}

% \bigskip

% \underline{Week 14, Case Study and Review:}
% A review of the prior weeks and applications to a
% new dataset. Second midterm on Thursday.




\end{document}





